\documentclass[12pt]{article}

\usepackage{amsmath}
\usepackage{amssymb}
\usepackage{stmaryrd}
\usepackage{booktabs}
\usepackage{mathpartir}
\usepackage[backend=bibtex, style=numeric]{biblatex}
\usepackage{hyperref}
\usepackage{soul}

% https://tex.stackexchange.com/questions/611251/in-circuitikz-multiwire-how-to-so-specify-label-placement
\usepackage[siunitx, RPvoltages, fulldiode, american voltages]{circuitikz}
\newcommand{\mwire}[3]{% args: position (0.1-0.9), label, final coordinate
    coordinate(tmp1) #3 coordinate(tmp2)
    (tmp1) -- ($(tmp1)!{#1-0.1}!(tmp2)$) to[multiwire=#2] ($(tmp1)!{#1+0.1}!(tmp2)$) -- (tmp2)
}

\usepackage{tikz}
\usetikzlibrary{positioning}
\usetikzlibrary{calc,decorations.markings}
\usetikzlibrary{quantikz2}
\usetikzlibrary{external}
\tikzexternalize

\usetikzlibrary{positioning}
\ctikzset{logic ports=ieee}

% \newcommand\ket[1]{
%   | #1 \rangle
% }

%   % \left\langle #1 \middle| #2 \right\rangle
% \newcommand\bra[1]{
%   \langle #1 |
% }

\newcommand\kket[1]{
  | #1 \rangle \rangle
}
% | #1 \rangle \kern-1.35pt \rangle


\begin{document}

\title{Lecture 2: Fourier transform}
\author{Lecturer: Rajendra Kumar\\ Scribe: Julin Shaji}
\date{June 16, 2025}
\maketitle

\section{Intro}
\begin{itemize}
\item DFT: Discrete Fourier Transform
\item FFT: Fast Fourier Transform
\item
  (We won't be talking about the classic Fourier transforms used
  for continuous signals).
\item
  An application of FFT = multiplication of two polynomials of the
  same degree.

\end{itemize}

\section{n-th roots of unity}
\begin{itemize}
\item $Z^n$ such that $(Z_n)^n = 1$
\item Examples:\\
% \begin{center}
  \begin{tabular}{cl}
    \toprule
    $n$  & $Z_n$ \\
    \midrule
   1st & $\{1\}$ \\
   2nd & $\{1, -1\}$ \\
   4th & $\{1, -1, i, -i\}$ \\
    \bottomrule
  \end{tabular}
% \end{center}
\end{itemize}

\section{Primitive roots of unity}
\begin{itemize}
\item There are many roots of unity
\item Special case: \emph{Primitive roots of unity}
\item $n-th$ primitive root of unity: $\omega_n$
\item A root of unity $Z_n$ is an $n-th$ primitive root of unity iff:
  \begin{itemize}
  \item $(Z_n)^n = 1$ (ie, must be a root of unity)
  \item $(Z_n)^j \neq 1, \forall j<n$
  \end{itemize}
\item Euler's formula:
\begin{mathpar}
 e^{2 \pi i \theta} = cos(2\pi\theta)  + i.sin(2\pi\theta)
\end{mathpar}
\end{itemize}
  
Equation for N-th primitive root of unity
  \begin{mathpar}
\omega_N = e^{-2 \pi i / N}
\\
\begin{array}{cl}
   & \omega_N^N \\
 = & (e^{-2 \pi i / N})^N \\
 = & e^{-2 \pi i} \\
 = & e^{2 \pi i (-1)} \\
 = & cos(2\pi (-1))  + i.sin(2\pi (-1)) \\
 = & cos(2\pi) - i.sin(2\pi) \\
 = & cos(0) - i.sin(0) \\
 = & 1 - i.0 \\
 = & 1 \\
\end{array}
  \end{mathpar}
  
\section{Fourier transform}
\begin{itemize}
\item A linear transformation $\Rightarrow$ a matrix
\item $v \to F.v = \hat{v}$
\item Input vector = an N-dimensional vector. $v \in \mathbb{R}^N$
  \begin{mathpar}
v = 
    \begin{bmatrix}
v_{11} \\      
v_{21} \\      
\vdots \\      
v_{n1} \\      
    \end{bmatrix}
  \end{mathpar}
\item k-th element of $\hat{v}$ is:
  \begin{mathpar}
\hat{v}_k = \sum\limits_{j=0}^{N-1} v_j e^{-2\pi i j k / N}
\\
\text{Euler's formula says:}
\\
\omega_N^j = e^{-2\pi i j / N} = cos(-2\pi j /N) + i.sin(-2\pi j/N)
\\
\text{Therefore,}
\\
\begin{array}{lcr}
\hat{v}_k & = & \sum\limits_{j=0}^{N-1} v_j e^{-2\pi i j k / N} \\
          & = & \sum\limits_{j=0}^{N-1} v_j \omega_N^{j.k} \\
\end{array}

  \end{mathpar}
\item What would the matrix $F$ for Fourier transform that can convert
  $v$ to $\hat{v}$ be?
  \begin{mathpar}
    \begin{bmatrix}
    ? & ? & \ldots & ? \\  
    ? & ? & \ldots & ? \\  
      &   & \vdots &   \\  
    ? & ? & \ldots & ? \\  
    \end{bmatrix}
    \begin{bmatrix}
    v_0 \\  
    v_1 \\  
    \vdots \\
    v_{N-1} \\  
    \end{bmatrix}
    =
    \begin{bmatrix}
    \hat{v_0} \\  
    \hat{v_1} \\  
    \vdots \\
    \hat{v_{N-1}} \\  
    \end{bmatrix}
  \end{mathpar}
\item As per the equation, first element of $\hat{v}$ (ie,
  $\hat{v_0}$) should be same sum of elements of input vector $v$
  $\Rightarrow$ first row of $\hat{v}$ is all ones.
\item Similarly, we can see that $\hat{v_1} =
  \sum\limits_{j=0}^{N-1}v_j w_N^j$
\item Leading us to figure out that:
  \begin{mathpar}
    F_N = 
  \frac{1}{\sqrt{N}}
    \begin{bmatrix}
     1 &        1   & 1          & \cdots &  1 \\ 
     1 & \omega_N   & \omega_N^2 & \cdots & \omega_N^{N-1} \\ 
     1 & \omega_N^2 & \omega_N^4 & \cdots & \omega_N^{2(N-1)} \\ 
       &            & \ldots     &        &    \\
     1 & \omega_N^{N-1} & \omega_N^{2(N-1)} & \cdots &  \omega_N^{(N-1)(N-1)} \\ 
    \end{bmatrix}
  \end{mathpar}
\end{itemize}

Example:

Find $F^3
  \begin{bmatrix}
    0 \\
    1 \\
    0 \\
  \end{bmatrix}$

\begin{mathpar}
  \begin{array}{cl}
    &
  F^3
  \begin{bmatrix}
    0 \\
    1 \\
    0 \\
  \end{bmatrix} \\
= &
  \frac{1}{\sqrt{3}}
  \begin{bmatrix}
    1 & \omega & \omega \\
    1 & \omega^2 & \omega^4 \\
    1 & \omega^3 & \omega^6 \\
  \end{bmatrix}
  \begin{bmatrix}
    0 \\
    1 \\
    0 \\
  \end{bmatrix} \\
= &
    \frac{1}{\sqrt{3}}
  \begin{bmatrix}
    \omega \\
    \omega^2 \\
    \omega^3 \\
  \end{bmatrix} \\
= &
    \frac{1}{\sqrt{3}}
  \begin{bmatrix}
    \omega \\
    \omega^2 \\
    1 \\
  \end{bmatrix} \\
  \end{array}
\end{mathpar}

\section{Properties of a unitary matrix}
\begin{itemize}
\item All column vectors are orthogonal to each other
\item All columns = Unit length (ie, the norm. Square root of sum of squares)
\end{itemize}

Is $F_N$ unitary?
\begin{itemize}
\item Yes.
\item Norm of each column = $\sqrt{N}$
\end{itemize}

\begin{mathpar}
  \begin{array}{cl}
   & Z^N-1 \\
 = & (Z-1)(Z-\omega) \cdots (Z-\omega^{N-1}) \\
 = & Z^N-(1+\omega+\omega^2+\cdots+\omega^{N-1}) Z^{N-1} + \cdots \\
\Rightarrow & (1+\omega+\omega^2+\cdots+\omega^{N-1}) = 0  (HOW??) \\
  \end{array}
\end{mathpar}


\section{Fast fourier transform}
\begin{itemize}
\item $F_N \Rightarrow$ matrix multiplication $\Rightarrow O(N^2)$
\item We can do better with FFT $\Rightarrow O(n logn)$
\item A recursive algorithm
\end{itemize}

Recall,
 
\begin{mathpar}
\hat{v_j} = \sum\limits_{k=0}^{N-1} v_k w_N^{j.k}
\end{mathpar}

Let us divide this into two equal parts. One for even $j$ and other
for odd $j$.

\begin{mathpar}
  \begin{array}{rcl}
\hat{v_j} & = & \sum\limits_{k is even}^{} v_k w_N^{j.k}
            + \omega^j \sum\limits_{k is odd}^{} v_k w_N^{j(k-1)} \\
 & = & \sum\limits_{k is even}^{} v_k w_{N/2}^{kj/2}
            + \omega^j \sum\limits_{k is odd}^{} v_k w_{N/2}^{j(k-1)/2} \\           
  \end{array}
\end{mathpar}

Thus we convert one DFT over N dimensions into two DFTs over N/2
dimensions each.


Time complexity:
\begin{itemize}
\item Every step halves the size of input being analyzed
\item Work needed to merge back results = 4N (how??)  
\item Size of an individual DFT at k-th recursion = $\frac{N}{2^k}$
\end{itemize}

\begin{mathpar}
  \begin{array}{rcl}
T(N) & = & 2T(N/2) * 4N \\
    & = & O(logN) * O(N) \\
    & = & O(N logN) \\
  \end{array}
\end{mathpar}


\section{Convolution of two vectors}
\begin{itemize}
\item Two vectors $a$ and $b$ both of dimension $N \in \mathbb{R}^N$
\item Result is an N-dimensional vector: $a * b$
\item i-th element of the result is given as:
  \begin{mathpar}
    (a * b)_i = \sum\limits_{j=0}^{N-1} a_i b_{(i-j)mod N}
  \end{mathpar}
\end{itemize}

\section{Polynomial multiplication with FFT}
\begin{itemize}
\item An application of Fourier transform
\item Input: 2 polynomials $p$ and $q$ each of degree $d$
  \begin{mathpar}
p(x) = \sum\limits_{i=0}^{d} a_i x^i

q(x) = \sum\limits_{i=0}^{d} b_i x^i
  \end{mathpar}
\item Output: Product of $p$ and $q$, which has degree $2d$
  \begin{mathpar}
    \begin{array}{rcl}
pq(x) & = & (a_0 b_0)x^0 + (a_0 b_1 + a_1 b_0)x^1 + \cdots  \\
      & = & \sum\limits_{i=0}^{2d} \left(\sum\limits_{j=0}^{i} a_j
            b_(i-j) \right) x^i \\
    \end{array}
  \end{mathpar}
\item How can we use FFT here?
\item Vector convolution operation helps.
\end{itemize}

Example:

\end{document}
