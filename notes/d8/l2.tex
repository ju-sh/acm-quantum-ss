\documentclass[12pt]{article}

\usepackage{amsmath}
\usepackage{amssymb}
\usepackage{stmaryrd}
\usepackage{booktabs}
\usepackage{mathpartir}
\usepackage[backend=bibtex, style=numeric]{biblatex}
\usepackage{hyperref}
\usepackage{soul}

% https://tex.stackexchange.com/questions/611251/in-circuitikz-multiwire-how-to-so-specify-label-placement
\usepackage[siunitx, RPvoltages, fulldiode, american voltages]{circuitikz}
\newcommand{\mwire}[3]{% args: position (0.1-0.9), label, final coordinate
    coordinate(tmp1) #3 coordinate(tmp2)
    (tmp1) -- ($(tmp1)!{#1-0.1}!(tmp2)$) to[multiwire=#2] ($(tmp1)!{#1+0.1}!(tmp2)$) -- (tmp2)
}

\usepackage{tikz}
\usetikzlibrary{positioning}
\usetikzlibrary{calc,decorations.markings}
\usetikzlibrary{quantikz2}
\usetikzlibrary{external}
\tikzexternalize

\usetikzlibrary{positioning}
\ctikzset{logic ports=ieee}

% \newcommand\ket[1]{
%   | #1 \rangle
% }

%   % \left\langle #1 \middle| #2 \right\rangle
% \newcommand\bra[1]{
%   \langle #1 |
% }

\newcommand\kket[1]{
  | #1 \rangle \rangle
}
% | #1 \rangle \kern-1.35pt \rangle


\begin{document}

\title{Lecture 2: Shor's algorithm}
\author{Lecturer: Dinesh Krishnamurthi\\ Scribe: Julin Shaji}
\date{June 17, 2025}
\maketitle

% \begin{center}
%   \begin{quantikz}
% \ket{x}   & \qwbundle{n} & \gate[2]{U_f} & \qwbundle{n} & \ket{x} \\
% \ket{0^n} & \qwbundle{m} &               & \qwbundle{m} & \ket{f(x)} \\
%   \end{quantikz}
% \end{center}

\url{https://people.eecs.berkeley.edu/~vazirani/f04quantum/notes/lec9.pdf}

\begin{itemize}
\item Finds prime factors of a number
\item Quantum algorithm by Peter Shor in 1994~\cite{shor1994algorithms}
\item Made possible via this path
  \begin{enumerate}
  \item Order finding problem
  \item 'Square-root finding problem'
  \item Factorization problem in \emph{polynomial time} with
    \emph{bounded error probability}
  \end{enumerate}
\item
Problem Statement:

Given an integer $M$, find prime number $p_i$ such that
\begin{mathpar}
  M = p_1^{e_1} + p_2^{e_2} + \cdots + p_k^{e_k}
\end{mathpar}
\end{itemize}

For simplicity, let us change the problem statement a bit and make the
following assumptions (without loss of generality):
\begin{itemize}
\item $M = pq$ where $p$ and $q$ are primes
\item It is sufficient to split $M$ into two numbers $a$ and $b$. We
  can repeat the algorithm recursively to find the factors in $log M$ time
\end{itemize}

\begin{itemize}
\item Classic prime factorization problems $\Rightarrow O(\sqrt{M})$
  (DBT: Which one is that? Sieve of Eratosthenes??)
\end{itemize}

\section{'Square root finding' problem}
Problem:
\begin{itemize}
\item Given:
  \begin{itemize}
  \item a composite number $N$
  \item $x$ such that $x$ is a non-trivial square root of $1$, $\bmod{N}$
  \end{itemize}
\item Wanted: Non-trivial factors of $N$
\item 
  Conditions to satisfy:
  \begin{itemize}
  \item $t^2 \equiv 1 \bmod{N}$
  \item $t \not\equiv 1 \bmod{N}$
  \item $t \not\equiv (-1) \bmod{N}$
  \end{itemize}
\end{itemize}

\begin{mathpar}
  \begin{array}{crcl}
            & t^2        & \equiv & 1 \bmod{N} \\
\Rightarrow & t^2-1      & \equiv & 0 \bmod{N} \\
\Rightarrow & (t+1)(t-1) & \equiv & 0 \bmod{N} \\
  \end{array}
\end{mathpar}

Since:

\begin{itemize}
  \item $t \not\equiv 1 \bmod{N}$
  \item $t \not\equiv (-1) \bmod{N}$
\end{itemize}

we can tell that:

\begin{itemize}
\item $1 < t < N-1$
\end{itemize}

Also, we know that

\begin{itemize}
\item $gcd(t+1, N)$ is a non-trivial factor of $N$
\item $gcd(t-1, N)$ is a non-trivial factor of $N$
\end{itemize}

Use Euclid's algorithm to find gcd and we get two non-trivial factors
of $N$.

\section{Order finding lemma}

Lemma description:
\begin{itemize}
\item Given
  \begin{itemize}
  \item $p$ is an odd prime
  \item $x$ is a uniformly distributed random variable
  \item $0 \le x < p$
  \end{itemize}
\item Then, ord(x) is even with probability half.
\end{itemize}

\section{Order finding algorithm}

\begin{itemize}
\item 
  In the last lecture, we saw a quantum algorithm to find the period of
  a given periodic function.
\item 
  The same technique can be used to find the order of an element $a$
  in a group $G$ = $ord_G(a)$
\end{itemize}



\end{document}
