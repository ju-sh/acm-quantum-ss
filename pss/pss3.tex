\documentclass[12pt]{exam}

\usepackage{amsmath}
\usepackage{amssymb}
\usepackage{stmaryrd}
\usepackage{booktabs}
\usepackage{mathpartir}
\usepackage[backend=bibtex, style=numeric]{biblatex}
\usepackage{hyperref}
\usepackage{soul}

% https://tex.stackexchange.com/questions/611251/in-circuitikz-multiwire-how-to-so-specify-label-placement
\usepackage[siunitx, RPvoltages, fulldiode, american voltages]{circuitikz}
\newcommand{\mwire}[3]{% args: position (0.1-0.9), label, final coordinate
    coordinate(tmp1) #3 coordinate(tmp2)
    (tmp1) -- ($(tmp1)!{#1-0.1}!(tmp2)$) to[multiwire=#2] ($(tmp1)!{#1+0.1}!(tmp2)$) -- (tmp2)
}

\usepackage{tikz}
\usetikzlibrary{positioning}
\usetikzlibrary{calc,decorations.markings}
\usetikzlibrary{quantikz2}
\usetikzlibrary{external}
\tikzexternalize

\usetikzlibrary{positioning}
\ctikzset{logic ports=ieee}

% \newcommand\ket[1]{
%   | #1 \rangle
% }

%   % \left\langle #1 \middle| #2 \right\rangle
% \newcommand\bra[1]{
%   \langle #1 |
% }

\newcommand\kket[1]{
  | #1 \rangle \rangle
}
% | #1 \rangle \kern-1.35pt \rangle


\title{Problem solving session 3}
\author{\href{https://github.com/ju-sh/acm-quantum-ss}{{\small ju-sh/acm-quantum-ss}}}
\date{17 June 2025}

\begin{document}

\maketitle
\printanswers

\begin{questions}

\question
  \label{q1}
  Consider an operator whose mapping is:
  \begin{itemize}
  \item $R_k\ket{0} = \ket{0}$ 
  \item $R_k\ket{1} = e^{2\pi i / 2^k} \ket{1}$ 
  \end{itemize}

  \begin{parts}
  \part
    Prove that $R_k$ is unitary
\begin{solution}
If a matrix is unitary, its complex conjugate would be its inverse.

  \begin{mathpar}
    R_k = 
    \begin{bmatrix}
      1 & 0 \\
      0 & e^{2\pi i / 2^k} \\
    \end{bmatrix}
    
    R_k^* =
    \begin{bmatrix}
      1 & 0 \\
      0 & e^{-2\pi i / 2^k} \\
    \end{bmatrix} \\
  \end{mathpar}
  
  \begin{align*}
    R_k * R_k^* &= 
    \begin{bmatrix}
      1 & 0 \\
      0 & e^{2\pi i / 2^k} \\
    \end{bmatrix} *
    \begin{bmatrix}
      1 & 0 \\
      0 & e^{-2\pi i / 2^k} \\
    \end{bmatrix} \\
    &= 
    % \begin{bmatrix}
    %   1.1 + 0.0 & 1.0 + 0.e^{-2\pi i / 2^k} \\
    %   0.1 + e^{2\pi i / 2^k}.0 & e^{2\pi i / 2^k}e^{-2\pi i / 2^k} \\
    % \end{bmatrix} \\
    \begin{bmatrix}
      1 & 0 \\
      0 & e^{2\pi i / 2^k}e^{-2\pi i / 2^k} \\
    \end{bmatrix} \\
    &= 
    \begin{bmatrix}
      1 & 0 \\
      0 & 1 \\
    \end{bmatrix} \\
    &= I
  \end{align*}
  
Therefore, complex conjugate of $R_k$ is its inverse.
Hence, $R_k$ is unitary.
  
\end{solution}

  \part
    Write down the eigen states, eigen values and corresponding phases
    of $R_k$.
\begin{solution}
Let:

\begin{itemize}
\item 
  Eigen state = $
  \begin{bmatrix}
    \alpha \\
    \beta  \\
  \end{bmatrix}$
\item 
  Eigen value = $\lambda$
\end{itemize}

Then,

\begin{align*}
R_k * 
  \begin{bmatrix}
    \alpha \\
    \beta  \\
  \end{bmatrix}
=& 
  \lambda 
  \begin{bmatrix}
    \alpha \\
    \beta  \\
  \end{bmatrix} \\
\begin{bmatrix}
  1 & 0 \\
  0 & e^{2\pi i / 2^k} \\ 
\end{bmatrix} 
 *
  \begin{bmatrix}
    \alpha \\
    \beta  \\
  \end{bmatrix}
=& 
  \lambda 
  \begin{bmatrix}
    \alpha \\
    \beta  \\
  \end{bmatrix} \\
  \begin{bmatrix}
    \alpha \\
    \beta(e^{2\pi i / 2^k})  \\
  \end{bmatrix}
=& 
  \lambda 
  \begin{bmatrix}
    \alpha \\
    \beta  \\
  \end{bmatrix} \\
\end{align*}

We know that $\alpha \neq 0$ and $\beta \neq 0$.

\begin{mathpar}
  \begin{array}{rcl}
  \alpha = \lambda \alpha & \Rightarrow & \lambda = 1  \\
  \beta(e^{2\pi i / 2^k}) = \lambda \beta & \Rightarrow & \lambda = e^{2\pi i / 2^k}
  \end{array}
\end{mathpar}

% \begin{itemize}
% \item $\alpha = \lambda \alpha \Rightarrow \lambda = 1$ 
% \item $\beta(e^{2\pi i / 2^k}) = \lambda \beta \Rightarrow \lambda = e^{2\pi i / 2^k}$ 
% \end{itemize}

Therefore, required answer is:

\begin{center}
\begin{tabular}{l|lll}
  \toprule
  & State & Eigen value & Phase \\
  \midrule
$R_k\ket{0}$ & $\ket{0}$ & 1 & 0 \\
$R_k\ket{1}$ & $\ket{1}$ & $e^{2\pi i / 2^k}$ & $e^{2\pi i / 2^k}$ \\
  \bottomrule
\end{tabular}
\end{center}

(Phase is the coefficient of the last qubit.)

\end{solution}
  \end{parts}
  
%%%%%%%%%%%%

\question
  Let:
  \begin{itemize}
  \item $c$ denote an n-bit integer
  \item $c = c_1 c_2 \cdots c_n$ be the bits of $c$
  \end{itemize}
  
  \begin{parts}
  \part
    Prove that
    \begin{mathpar}
e^{2\pi i c / 2^n} \ket{c}
=
\bigotimes\limits_{j=1}^{n} e^{2\pi i c_j / 2^j} \ket{c_j}
    \end{mathpar}
    
Fun fact:
This shows how to push a value from the state to the phase of an
amplitude.

\begin{solution}
$c$ is an integer. Expressed in bit representation.

\begin{mathpar}
  \begin{array}{rcl}
  c & = & c_1 \cdot c_2 \cdot c_3 \cdots c_n \\
    & = & c_1 2^{n-1} \cdot c_2 2^{n-2} \cdot c_3 2^{n-3} \cdots c_n 2^{n-n} \\
  \end{array}
\end{mathpar}

So,

  \begin{align*}
 & e^{2\pi i c / 2^n} \ket{c_i} \\
 & \Rightarrow
   e^{2\pi i c_i 2^{n-i} / 2^n} \ket{c_i} \\
 & =
   e^{2\pi i(c_1 2^{n-1} + c_2 2^{n-2} + \cdots + \c_n 2^0)/2^n}
   \ket{c_1 c_2 \cdots c_n} \\
 & =
   \prod\limits_{j=1}^{n} 
   \left[ e^{2\pi i c_j / 2^j} \right] \ket{c_1 c_2 \cdots c_n} \\
 & =
   \bigotimes\limits_{j=1}^{n} e^{2\pi i c_j / 2^j} \ket{c_j}
  \end{align*}
\end{solution}

  \part
    Use $R_k$ gates from last question to implement 
    \begin{mathpar}
\ket{c} \mapsto e^{2\pi i c / 2^n}\ket{c}
\end{mathpar}
    
\begin{solution}
We need a multiplication by $e^{2\pi i c / 2^n}$ on each input qubit
where $c_j \in \{0, 1\}$, which is exactly what $R_n$ does.
So, use $R_n$ gate on each qubit.

\begin{center}
  \begin{quantikz}
    \lstick{\ket{c_1}} & \gate{R_n} & \rstick[4]{$\ \ e^{2\pi i c / 2^n}$\ket{c}} \\
    \lstick{\ket{c_2}} & \gate{R_n} &  \\
      \vdots & \vdots  & \vdots \\
    \lstick{\ket{c_n}} & \gate{R_n} & 
  \end{quantikz}
\end{center}

\end{solution}
  \end{parts}
  

%%%%%%%%%%%%

\question
  Let:
  \begin{itemize}
  \item $a$ be an n-bit integer and $b$ denote $(a+1) \pmod{n}$
  \item $\ket{\hat{a}}$ denote QFT($\ket{a}$)
  \item $\ket{\hat{b}}$ denote QFT($\ket{b}$)
  \end{itemize}

  \begin{parts}
  \part
    Design a quantum circuit for $\ket{\hat{a}} \mapsto \ket{\hat{b}}$
\begin{solution}
The equation for finding QFT (as tensor product) is:

\begin{align*}
\ket{\hat{a}}
 % &= \bigotimes\limits_{j=1}^{n} \frac{1}{\sqrt{2}}
 %     \left(
 %       \ket{0} +
 %       e^{2\pi i a (0.k_{n+j-1} \cdots k_{n})} \ket{1}
 %     \right) \\
 &= \bigotimes\limits_{j=1}^{n} \frac{1}{\sqrt{2}}
     \left(
       \ket{0} +
       e^{2\pi i a / 2^j} \ket{1}
     \right) \\
\end{align*}

So, t-th qubit of
\begin{itemize}
\item QFT($a$): $
    \frac{1}{\sqrt{2}}
       \left(
         \ket{0} +
         e^{2\pi i a / 2^t} \ket{1}
       \right)$
\item QFT($b$):
  \begin{align*}
      &  
   \frac{1}{\sqrt{2}}
       \left(
         \ket{0} +
         e^{2\pi i b / 2^t} \ket{1}
       \right) \\
    = & 
   \frac{1}{\sqrt{2}}
       \left(
         \ket{0} +
         e^{2\pi i (a+1) / 2^t} \ket{1}
       \right) \\
    = & 
   \frac{1}{\sqrt{2}}
       \left(
         \ket{0} +
         e^{2\pi i a / 2^t} e^{2\pi i/ 2^t} \ket{1}
       \right) \\
    = & 
    e^{2\pi i/ 2^t}
   \frac{1}{\sqrt{2}}
       \left(
         \ket{0} +
         e^{2\pi i a / 2^t} \ket{1}
       \right) \\
    = & 
    R_t
    \left[
   \frac{1}{\sqrt{2}}
       \left(
         \ket{0} +
         e^{2\pi i a / 2^t} \ket{1}
       \right)
    \right] \\
    = & 
    R_t \ket{\hat{a}} \\
  \end{align*}
\end{itemize}

ie, we just need to apply the $R_t$ gate (from question~\ref{q1})
on each of the input qubits.

\begin{center}
  \begin{quantikz}
    \lstick{\ket{\hat{a}_1}} & \gate{R_1} & \rstick{\ket{\hat{b}_1}} \\
    \lstick{\ket{\hat{a}_2}} & \gate{R_2} & \rstick{\ket{\hat{b}_2}} \\
      \vdots & \vdots  & \vdots \\
    \lstick{\ket{\hat{a}_n}} & \gate{R_n} & \rstick{\ket{\hat{b}_n}} \\
  \end{quantikz}
\end{center}
\end{solution}

  \part
    Design a quantum circuit using QFT for $\ket{a} \mapsto \ket{(a+1) \pmod{n}}$
\begin{solution}
  \begin{mathpar}
    \begin{array}{crcl}
            & \ket{a} & \mapsto & \ket{b} \\
\Rightarrow & IQFT(\ket{\hat{a}}) & \mapsto & IQFT(\ket{\hat{b}}) \\
    \end{array}
  \end{mathpar}

So, we can use most of the circuitry from the first sub-question.

\begin{center}
  \begin{quantikz}%[slice all]
    \lstick{\ket{a_1}}
      & \gate{QFT}
        \slice[label style={pos=1, anchor=north}]{\ket{\hat{a}_i}}
      & \gate{R_1}
        \slice[label style={pos=1, anchor=north}]{\ket{\hat{b}_i}}
      & \gate{IQFT}
      & \rstick{\ket{b_1}} \\
    \lstick{\ket{a_2}}
      & \gate{QFT}
      & \gate{R_2}
      & \gate{IQFT}
      & \rstick{\ket{b_2}} \\
    \vdots
      & \vdots
      & \vdots
      & \vdots
      & \vdots \\
    \lstick{\ket{a_n}}
      & \gate{QFT}
      & \gate{R_n}
      & \gate{IQFT}
      & \rstick{\ket{b_n}}
  \end{quantikz}
\end{center}
\end{solution}
  \end{parts}
\end{questions}

\end{document}

