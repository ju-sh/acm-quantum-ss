\documentclass[12pt]{exam}

\usepackage{amsmath}
\usepackage{amssymb}
\usepackage{stmaryrd}
\usepackage{booktabs}
\usepackage{mathpartir}
\usepackage[backend=bibtex, style=numeric]{biblatex}
\usepackage{hyperref}
\usepackage{soul}

% https://tex.stackexchange.com/questions/611251/in-circuitikz-multiwire-how-to-so-specify-label-placement
\usepackage[siunitx, RPvoltages, fulldiode, american voltages]{circuitikz}
\newcommand{\mwire}[3]{% args: position (0.1-0.9), label, final coordinate
    coordinate(tmp1) #3 coordinate(tmp2)
    (tmp1) -- ($(tmp1)!{#1-0.1}!(tmp2)$) to[multiwire=#2] ($(tmp1)!{#1+0.1}!(tmp2)$) -- (tmp2)
}

\usepackage{tikz}
\usetikzlibrary{positioning}
\usetikzlibrary{calc,decorations.markings}
\usetikzlibrary{quantikz2}
\usetikzlibrary{external}
\tikzexternalize

\usetikzlibrary{positioning}
\ctikzset{logic ports=ieee}

% \newcommand\ket[1]{
%   | #1 \rangle
% }

%   % \left\langle #1 \middle| #2 \right\rangle
% \newcommand\bra[1]{
%   \langle #1 |
% }

\newcommand\kket[1]{
  | #1 \rangle \rangle
}
% | #1 \rangle \kern-1.35pt \rangle


\title{Problem solving session 4}
\author{\href{https://github.com/ju-sh/acm-quantum-ss}{{\small ju-sh/acm-quantum-ss}}}
\date{18 June 2025}

\begin{document}

\maketitle
\printanswers

  Consider the following measurement operator acting on a single qubit
  $\ket{\psi}$.
  
\begin{center}
  \begin{quantikz}
    \lstick{\ket{\psi}}
    & \meter{X_B}
    & 
  \end{quantikz}
\end{center}


\begin{mathpar}
X_B(\ket{\psi}) =
  \begin{cases}
  \ket{0} & \text{nothing happens} \\ 
  \ket{1} & \text{bomb explodes}
  \end{cases}
\end{mathpar}

In general,

\begin{center}
  \begin{quantikz}
    \lstick{$\alpha$\ket{0} + $\beta$\ket{1}}
    & \meter{X_B}
    & 
  \end{quantikz}%
\end{center}

\begin{mathpar}
X_B(\alpha\ket{0} + \beta\ket{1}) =
  \begin{cases}
  \ket{0} & \text{nothing happens with probability } |\alpha|^2\\ 
  \ket{1} & \text{bomb explodes with probability } |\beta|^2
  \end{cases}
\end{mathpar}

It is guaranteed that $\ket{\psi}$ is either $\ket{0}$ or $\ket{1}$.

Your goal is to identify if (without dying from a bomb explosion):
\begin{itemize}
\item $\ket{\psi} = 0$ (not bomb)
\item $\ket{\psi} = 1$ (bomb)
\end{itemize}

\begin{questions}

\question
  What happens when you measure $\ket{\psi}$ using $X_B$?
  
  \begin{solution}
Bomb explodes or doesn't. Each with probability $\frac{1}{2}$.
  \end{solution}

\question
  Consider the following circuit.

\begin{center}
  \begin{quantikz}
    \lstick{\ket{0}}
      \slice[label style={pos=1, anchor=north}]{1}
    & \gate{H}
      \slice[label style={pos=1, anchor=north}]{2}
    & \ctrl{1}
      \slice[label style={pos=1, anchor=north}]{3}
    &  
      \slice[label style={pos=1, anchor=north}]{4}
    & \meter{} \\
    \lstick{\ket{\psi}}
    & 
    & \gate{X}
    & \meter{X_B}
    &  
  \end{quantikz}%
\end{center}

  Write down the states at 1,2,3,4

  \begin{solution}
States would be:
\begin{itemize}
\item 1: $\ket{0}\ket{\psi}$
\item 2: Hadamard on qubit-1
  \begin{align*}
H\ket{0}\ket{\psi}
  &=
  \frac{1}{\sqrt{2}}(\ket{0} + \ket{1})\ket{\psi} \\
  &=
  \frac{1}{\sqrt{2}}(\ket{0\psi} + \ket{1\psi}) \\
  \end{align*}
\item 3: Apply controlled NOT gate
  \begin{align*}
\frac{1}{\sqrt{2}}(\ket{0\psi} + \ket{1\psi})
  &\Rightarrow
  \frac{1}{\sqrt{2}}(\ket{0\psi} + \ket{1\bar{\psi}}) \\
  \end{align*}
\item 4:
  Measure qubit-2. It will collapse to a particular state.
  $0$ or $1$ with $1/2$ probability 
  
  \begin{itemize}
  \item If $\ket{\psi} = \ket{0}$,
    \begin{align*}
 & \frac{1}{\sqrt{2}}(\ket{0\psi} + \ket{1\bar{\psi}}) \\
 &= \frac{1}{\sqrt{2}}(\ket{00} + \ket{11}) \\
    \end{align*}
    %
    \begin{itemize}
    \item Collapse to 0 $\Rightarrow \ket{00}$ 
    \item Collapse to 1 $\Rightarrow \ket{11}$ 
    \end{itemize}
  \item If $\ket{\psi} = \ket{1}$,
    \begin{align*}
 & \frac{1}{\sqrt{2}}(\ket{0\psi} + \ket{1\bar{\psi}}) \\
 &= \frac{1}{\sqrt{2}}(\ket{01} + \ket{10}) \\
    \end{align*}
    %
    \begin{itemize}
    \item Collapse to 0 $\Rightarrow \ket{10}$ 
    \item Collapse to 1 $\Rightarrow \ket{01}$ 
    \end{itemize}
  \end{itemize}
  
\item 5:
  Measure qubit-1. Qubit-2 has already collapsed.
  Qubit-1 is entangled $\Rightarrow$ Qubit-1 too can be found.
  
  \begin{center}
    \begin{tabular}{ccl}
      \toprule
      State & Qubit-1 & Path \\
      \midrule
      $\ket{00}$ & 0 & $\ket{\psi} = \ket{0} \text{then } q_2 \mapsto 0$ \\
      $\ket{11}$ & 1 & $\ket{\psi} = \ket{0} \text{then } q_2 \mapsto 1$ \\
      $\ket{10}$ & 1 & $\ket{\psi} = \ket{1} \text{then } q_2 \mapsto 0$ \\
      $\ket{01}$ & 0 & $\ket{\psi} = \ket{1} \text{then } q_2 \mapsto 1$ \\
      \bottomrule
    \end{tabular}
  \end{center}

 %  \begin{itemize}
 %  \item If $\ket{\psi} = \ket{0}$,
 %    \begin{align*}
 % & \frac{1}{\sqrt{2}}(\ket{00} + \ket{11}) \\
 % &=  \\
 %    \end{align*}
 %  \item If $\ket{\psi} = \ket{1}$,
 %    \begin{align*}
 % & \frac{1}{\sqrt{2}}(\ket{0\psi} + \ket{1\bar{\psi}}) \\
 % &=  \\
 %    \end{align*}
 %  \end{itemize}
\end{itemize}
  \end{solution}
  
\question
We have an algorithm $A$:

\begin{enumerate}
\item Perform the operations in the above circuit
\item If the bomb explodes, then there is no output (person operating
  the circuit died..)
\item If it did not explode and the first register is:
  \begin{itemize}
  \item $\ket{0}$ then output ``No bomb''
  \item $\ket{1}$ then output ``Bomb''
  \end{itemize}
\end{enumerate}

Calculate the following probabilities.

\begin{parts}
\part
  Suppose $\ket{\psi} = \ket{1}$ (ie, ``Bomb'' output), compute
  Pr(Algorithm A outputs ``No bomb'').

\begin{solution}
  \begin{mathpar}
    \begin{array}{crcl}
            & \ket{\psi} & = & \ket{1} \\
\Rightarrow & \ket{\psi} & = & \frac{1}{\sqrt{2}}(\ket{01} - \ket{10})
      \text{(From step 3)}\\
    \end{array}
  \end{mathpar}
  
Measure with $X_B$.
\begin{itemize}
\item Bomb explode with probability $\frac{1}{2}$
  \begin{itemize}
  \item $\Rightarrow$ No output at all
  \item q2 collapses to $\ket{0}$
  \item Therefore, $\ket{\psi} = \ket{01}$
  \end{itemize}
\item No explosion with probability $\frac{1}{2}$
  \begin{itemize}
  \item An output still possible
  \item q2 collapses to $\ket{1}$
  \item Therefore, $\ket{\psi} = \ket{10}$
  \item $\Rightarrow$ output is ``Bomb''
  \end{itemize}
\end{itemize}

So, Pr(``No bomb'') = 0
\end{solution}

\part
  Suppose $\ket{\psi} = \ket{0}$ (ie, ``No bomb'' output), compute
  Pr(Algorithm A outputs ``Bomb'').

\begin{solution}
  \begin{mathpar}
    \begin{array}{crcl}
            & \ket{\psi} & = & \ket{0} \\
\Rightarrow & \ket{\psi} & = & \frac{1}{\sqrt{2}}(\ket{00} + \ket{11})
      \text{(From step 3)}\\
    \end{array}
  \end{mathpar}

Measure with $X_B$.
\begin{itemize}
\item Bomb explode with probability $\frac{1}{2}$
  \begin{itemize}
  \item $\Rightarrow$ No output at all
  \item q2 collapses to $\ket{1}$
  \item Therefore, $\ket{\psi} = \ket{11}$
  \end{itemize}
\item No explosion with probability $\frac{1}{2}$
  \begin{itemize}
  \item An output still possible
  \item q2 collapses to $\ket{0}$
  \item Therefore, $\ket{\psi} = \ket{00}$
  \item $\Rightarrow$ output is ``No Bomb''
  \end{itemize}
\end{itemize}

So, Pr(``Bomb'') = 0

\end{solution}

\part
  \st{The error probability of algorithm $A$ is the maximum of the
  probabilities computed in part (a) and part (b).}
  Find maximum of the above two probabilities.

\begin{solution}
  \begin{mathpar}
max(0, 0) = 0 
  \end{mathpar}
\end{solution}

\part
  The probability of explosion for algorithm $A$ is the maximum of
  probabilities of explosion when $\ket{\psi} = \ket{0}$ and 
  $\ket{\psi} = \ket{1}$.
  Compute the maximum probability of explosion for $A$.
  
\begin{solution}
  \begin{itemize}
  \item Pr(``Bomb when $\ket{\psi}=\ket{0}$) = $0$
  \item Pr(``Bomb when $\ket{\psi}=\ket{1}$) = $\frac{1}{2}$
  \end{itemize}
  
Therefore:
\begin{mathpar}
  max(0, \frac{1}{2}) = \frac{1}{2} 
\end{mathpar}
\end{solution}

\end{parts}
\end{questions}

\end{document}

