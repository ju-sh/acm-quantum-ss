\documentclass[12pt]{exam}

\usepackage{amsmath}
\usepackage{amssymb}
\usepackage{stmaryrd}
\usepackage{booktabs}
\usepackage{mathpartir}
\usepackage[backend=bibtex, style=numeric]{biblatex}
\usepackage{hyperref}
\usepackage{soul}

% https://tex.stackexchange.com/questions/611251/in-circuitikz-multiwire-how-to-so-specify-label-placement
\usepackage[siunitx, RPvoltages, fulldiode, american voltages]{circuitikz}
\newcommand{\mwire}[3]{% args: position (0.1-0.9), label, final coordinate
    coordinate(tmp1) #3 coordinate(tmp2)
    (tmp1) -- ($(tmp1)!{#1-0.1}!(tmp2)$) to[multiwire=#2] ($(tmp1)!{#1+0.1}!(tmp2)$) -- (tmp2)
}

\usepackage{tikz}
\usetikzlibrary{positioning}
\usetikzlibrary{calc,decorations.markings}
\usetikzlibrary{quantikz2}
\usetikzlibrary{external}
\tikzexternalize

\usetikzlibrary{positioning}
\ctikzset{logic ports=ieee}

% \newcommand\ket[1]{
%   | #1 \rangle
% }

%   % \left\langle #1 \middle| #2 \right\rangle
% \newcommand\bra[1]{
%   \langle #1 |
% }

\newcommand\kket[1]{
  | #1 \rangle \rangle
}
% | #1 \rangle \kern-1.35pt \rangle


\title{Problem solving session 5}
\author{\tiny{JS}}
\date{19 June 2025}

\begin{document}

\maketitle
\printanswers

\begin{questions}
  \question
  \label{q0}
  Recall the Fredkin gate (Controlled swap gate).
  
  \begin{mathpar}
    F(\ket{0}\ket{ab}) = \ket{0}\ket{??} \\ 
    F(\ket{1}\ket{ab}) = \ket{1}\ket{??} \\ 
  \end{mathpar}
  
  \begin{solution}
  Fredkin gate swaps 2nd and 3rd qubits if the 1st qubit is 1.

  \begin{mathpar}
    F(\ket{0}\ket{ab}) = \ket{0}\ket{ab} \\ 
    F(\ket{1}\ket{ab}) = \ket{1}\ket{ba} \\ 
  \end{mathpar}
  \end{solution}
  
  \question
  \label{q1}
  Suppose you are give two 1-qubit states $\ket{\psi_1}$ and
  $\ket{\psi_2}$ via unitaries $U_1$ and $U_2$.
  
  The goal is to test if the states $\ket{\psi_1}$ and $\ket{\psi_2}$
  are same or different.
  
  Consider the following circuit (called \emph{swap test}):

\begin{center}
  \begin{quantikz}
    \lstick{\ket{0}}
    & \slice[label style={pos=1, anchor=north}]{1}
    & \gate{H}
      \slice[label style={pos=1, anchor=north}]{2}
    &  
    & \ctrl{1}
      \gategroup[%
        3,
        steps=1,
        style={
          dashed,
          rounded corners,
          fill=blue!20,
          inner xsep=2pt
        },
        background,
        label style={
          label position=above,
          anchor=north,
          yshift=0.5cm
        }
      ]{{\small Fredkin}}
    & \slice[label style={pos=1, anchor=north}]{3}
    & \gate{H}
      \slice[label style={pos=1, anchor=north}]{4}
    & \meter{}
      \slice[label style={pos=1, anchor=north}]{5}
    & \\
    \lstick{\ket{\psi_1}}
    & \qwbundle{1} 
    &  
    &  
    & \swap{1}
    &
    &
    &
    & \\
    \lstick{\ket{\psi_2}}
    & \qwbundle{1} 
    &
    &  
    & \targX{}
    &
    &
    &
    &
  \end{quantikz}%
\end{center}

Write down the states at 1, 2, 3, 4 and 5.

\begin{solution}
  \begin{itemize}
\item 1: $\ket{0 \psi_1 \psi_2}$
\item 2: Hadamard on qubit-1
  \begin{align*}
H\ket{0} \otimes \ket{\psi_1 \psi_2}
  &=
  \ket{+} \otimes \ket{\psi_1 \psi_2} \\
  &=
  \frac{1}{\sqrt{2}}\left( \ket{0} + \ket{1} \right) \ket{\psi_1 \psi_2} \\
  &=
  \frac{1}{\sqrt{2}}\left( \ket{0\psi_1 \psi_2} + \ket{1\psi_1 \psi_2} \right) \\
  \end{align*}
\item 3: CSWAP on qubits 2 and 3, controlled by qubit-1
  \begin{align*}
\frac{1}{\sqrt{2}}\left( \ket{0\psi_1 \psi_2} + \ket{1\psi_1 \psi_2} \right)
  &\Rightarrow
\frac{1}{\sqrt{2}}\left( \ket{0\psi_1 \psi_2} + \ket{1\psi_2 \psi_1} \right)
  \end{align*}
\item 4: Hadamard on qubit-1 (again)
  \begin{align*}
\frac{1}{\sqrt{2}}\left( \ket{0\psi_1 \psi_2} + \ket{1\psi_2 \psi_1} \right)
    &=
\frac{1}{\sqrt{2}}
  \left(
      H\ket{0}\ket{\psi_1 \psi_2}
      +
      H\ket{1}\ket{\psi_2 \psi_1}
  \right) \\
    &=
\frac{1}{\sqrt{2}}
  \left(
      \ket{+}\ket{\psi_1 \psi_2}
      +
      \ket{-}\ket{\psi_2 \psi_1}
  \right) \\
    &=
\frac{1}{2}
  \left(
      (\ket{0}+\ket{1})\ket{\psi_1 \psi_2}
      +
      (\ket{0}-\ket{1})\ket{\psi_2 \psi_1}
  \right) \\
    &=
\frac{1}{2}
  \left(
        \ket{0\psi_1 \psi_2}
      + \ket{1\psi_1 \psi_2}
      + \ket{0\psi_2 \psi_1}
      - \ket{1\psi_2 \psi_1}
  \right) \\
    &=
\frac{1}{2}\ket{0}
  \left(
    \ket{\psi_1 \psi_2} + \ket{\psi_2 \psi_1}
  \right)
+
\frac{1}{2}\ket{1}
  \left(
    \ket{\psi_1 \psi_2} - \ket{\psi_2 \psi_1}
  \right)
  \end{align*}
\item 5: Measure qubit-1 (the first register) $\Rightarrow$
  qubit-1 collapses to $\ket{0}$ or $\ket{1}$.
% 
%   \begin{itemize}
%   \item $q1 = \ket{0} \Rightarrow 
% \frac{1}{\sqrt{2}}
%   \left(
%     \ket{\psi_1 \psi_2} + \ket{\psi_2 \psi_1}
%   \right)$
%   \item $q1 = \ket{1} \Rightarrow 
% \frac{1}{\sqrt{2}}
%   \left(
%     \ket{\psi_1 \psi_2} + \ket{\psi_2 \psi_1}
%   \right)$
%   \end{itemize}
%   That means state is going to end up being $
% \frac{1}{\sqrt{2}}
%   \left(
%     \ket{\psi_1 \psi_2} + \ket{\psi_2 \psi_1}
%   \right)$ anyway ??
%   Not sure if this part is correct. Partial measurement ??
  \end{itemize}
\end{solution}

  \question
  \label{q2}
Compute the probability of seeing $\ket{1}$ after the measurement in
register 1.

(Hint: It is a function of $|\braket{\psi_2}{\psi_1}|$.)

\begin{solution}
Before measuring qubit-1, the state is:

\begin{mathpar}
\frac{1}{2}\ket{0}
  \left(
    \ket{\psi_1 \psi_2} + \ket{\psi_2 \psi_1}
  \right)
+
\frac{1}{2}\ket{1}
  \left(
    \ket{\psi_1 \psi_2} - \ket{\psi_2 \psi_1}
  \right)
\end{mathpar}

\emph{Partial measurement} needed. Measure specific qubits, while leaving the
remaining qubits in the entangled state itself.

\begin{itemize}
\item Qubit-1 collapses to $\ket{1}$
\item Probability of this happening is square of its amplitude
\item Let this amplitude = $\phi$
  \begin{mathpar}
\phi = 
\frac{1}{2}
  \left(
    \ket{\psi_1 \psi_2} - \ket{\psi_2 \psi_1}
  \right)
  \end{mathpar}
\item Required probability is given as:
  \begin{align*}
\lVert \ket{\phi} \rVert^2
    &=
 |\braket{\phi}{\phi}| \\
    &=
\frac{1}{2}
(\bra{\psi_1}\bra{\psi_2} - \bra{\psi_2}\bra{\psi_1})
\cdot
\frac{1}{2}
(\ket{\psi_1}\ket{\psi_2} - \ket{\psi_2}\ket{\psi_1}) \\
    &=
\frac{1}{4}
\left(
  \bra{\psi_1}\braket{\psi_2}{\psi_1}\ket{\psi_2}
- \bra{\psi_1}\braket{\psi_2}{\psi_2}\ket{\psi_1}
- \bra{\psi_2}\braket{\psi_1}{\psi_1}\ket{\psi_2}
+ \bra{\psi_2}\braket{\psi_1}{\psi_2}\ket{\psi_1}
\right) \\
    &=
\frac{1}{4}
\left(
  \bra{\psi_1}\braket{\psi_2}{\psi_1}\ket{\psi_2}
- \braket{\psi_1}{\psi_1}
- \braket{\psi_2}{\psi_2}
+ \bra{\psi_2}\braket{\psi_1}{\psi_2}\ket{\psi_1}
\right) \\
    &=
\frac{1}{4}
\left(
\bra{\psi_1}
\braket{\psi_2}{\psi_1}
\ket{\psi_2}
-1 -1
+
\bra{\psi_2}
\braket{\psi_1}{\psi_2}
\ket{\psi_1}
\right) \\
    &=
\frac{1}{4}
\left(
\bra{\psi_1}
\braket{\psi_2}{\psi_1}
\ket{\psi_2}
-2
+
\bra{\psi_2}
\braket{\psi_1}{\psi_2}
\ket{\psi_1}
\right) \\
  \end{align*}
  
$\braket{\psi_2}{\psi_1}$ and $\braket{\psi_1}{\psi_2}$ are scalars
since the operation is inner product.
So, we can move them out.

\begin{align*}
  &=
\frac{1}{4}
\left(
\braket{\psi_2}{\psi_1}
\braket{\psi_1}{\psi_2}
-2
+
\braket{\psi_1}{\psi_2}
\braket{\psi_2}{\psi_1}
\right) \\
\end{align*}

% Since $\braket{\psi_1}{\psi_2} = \braket{\psi_2}{\psi_1}$,

% \begin{align*}
%   &=
% \frac{1}{4}
% \left(
% \braket{\psi_1}{\psi_2}
% \braket{\psi_1}{\psi_2}
% -2
% +
% \braket{\psi_1}{\psi_2}
% \braket{\psi_1}{\psi_2}
% \right) \\
%   &=
% \frac{1}{4}
% \left(
% \braket{\psi_1}{\psi_2}
% \braket{\psi_1}{\psi_2}
% -2
% +
% \braket{\psi_1}{\psi_2}
% \braket{\psi_1}{\psi_2}
% \right) \\
% \end{align*}

\end{itemize}

\end{solution}

  \question
  Let $D_1$ and $D_2$ be two 1-qubit operators defined as follows:
  
  \begin{mathpar}
    \begin{array}{rcl}
D_1\ket{0} & = & \sqrt{p_0}\ket{0} + \sqrt{1-p_0}\ket{1} \\
D_1\ket{1} & = & \sqrt{p_1}\ket{0} + \sqrt{1-p_1}\ket{1} \\
    \end{array}
  \end{mathpar}
  
and
  
  \begin{mathpar}
    \begin{array}{rcl}
D_2\ket{0} & = & \sqrt{q_0}\ket{0} + \sqrt{1-q_0}\ket{1} \\
D_2\ket{1} & = & \sqrt{q_1}\ket{0} + \sqrt{1-q_1}\ket{1} \\
    \end{array}
  \end{mathpar}
  
Note that:
\begin{itemize}
\item $D_1$ encodes a distribution $0 \mapsto p_0$ and $1 \mapsto p_1$
\item $D_2$ encodes a distribution $0 \mapsto q_0$ and $1 \mapsto q_1$
\end{itemize}

\begin{parts}
  \part
Suppose that you are given two states

\begin{mathpar}
  \ket{\psi_1} = \frac{D_1\ket{0} + D_1\ket{1}}{\sqrt{2}}

  \ket{\psi_2} = \frac{D_2\ket{0} + D_2\ket{1}}{\sqrt{2}}
\end{mathpar}

Use the information from question~\ref{q1} and question~\ref{q2} to
distinguish the two distributions $D_1$ and $D_2$.

\begin{solution}

\end{solution}

  \part
Now suppose that you are just given the operations $D_1$ and $D_2$ and
not the states $\ket{\psi_1}$ and $\ket{\psi_2}$.\\
How can you distinguish the two distributions in this case?

\begin{solution}
  
\end{solution}
\end{parts}

\end{questions}

\end{document}

