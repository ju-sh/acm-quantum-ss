\documentclass[12pt]{exam}

\input{../config.tex}

\title{Quiz 5}
\author{\href{https://github.com/ju-sh/acm-quantum-ss}{{\small ju-sh/acm-quantum-ss}}}
\date{11 June 2025}

\begin{document}
%\url{https://acm-qc-summer-school.gitlab.io/assets/pdfs/quiz5.pdf}

\maketitle
\printanswers

\begin{questions}

\question
Consider the following circuit.

\begin{center}
  \begin{quantikz}
    \lstick{\ket{x}} & \gate[3]{\text{Toffoli}} & \rstick{\ket{x}} \\
    \lstick{\ket{y}} &                          & \rstick{\ket{y}} \\
    \lstick{\ket{-}} &                          & 
  \end{quantikz}
\end{center}

Let $\ket{-}$ be equal to the state $H\ket{1}$. Let $a_{00}$,
$a_{01}$, $a_{10}$, $a_{11}$ be amplitudes such that,

\begin{mathpar}
T(\ket{00}_{xy} \ket{-}) = a_{00} \ket{00}_{xy} \ket{-} \\
T(\ket{01}_{xy} \ket{-}) = a_{01} \ket{01}_{xy} \ket{-} \\
T(\ket{10}_{xy} \ket{-}) = a_{10} \ket{10}_{xy} \ket{-} \\
T(\ket{11}_{xy} \ket{-}) = a_{11} \ket{11}_{xy} \ket{-} \\
\end{mathpar}

Write down the values of $a_{00}$, $a_{01}$, $a_{10}$, $a_{11}$.

\begin{solution}
Toffoli gate is as follows:  

\begin{center}
  \begin{quantikz}
    \lstick{$x$} & \gate[3]{\text{Toffoli}} & \rstick{$x$} \\
    \lstick{$y$} &                          & \rstick{$y$} \\
    \lstick{$b$} &                          & \rstick{$b \oplus xy$}
  \end{quantikz}
\end{center}

It's like a 'CCNOT', a NOT gate with two controls. 3rd output qubit is
flipped when both other qubits are 1.
Otherwise input and output are same.

Therefore,

\begin{itemize}
\item $T(\ket{00}_{xy} \ket{-}) = 1 \cdot \ket{00}_{xy} \ket{-}$
\item $T(\ket{01}_{xy} \ket{-}) = 1 \cdot \ket{01}_{xy} \ket{-}$
\item $T(\ket{10}_{xy} \ket{-}) = 1 \cdot \ket{10}_{xy} \ket{-}$
\end{itemize}

When the NOT gate is active,

\begin{mathpar}
  \begin{array}{crcl}
            &  \ket{11}\ket{-}   &=& \ket{11}\frac{1}{\sqrt{2}}(\ket{0} - \ket{1}) \\
  &&& \\
\Rightarrow & T(\ket{11}\ket{-}) &=& \ket{11}\frac{1}{\sqrt{2}}(\ket{1} - \ket{0}) \\
  &&=& -\ket{11}\frac{1}{\sqrt{2}}(\ket{0} - \ket{1}) \\
  &&=& -\ket{11}\ket{-} \\
  \end{array}
\end{mathpar}

So the required values are:
\begin{tabular}{cc}
  \toprule
Name & Value \\
  \midrule
$a_{00}$ &  1 \\
$a_{01}$ &  1 \\
$a_{10}$ &  1 \\
$a_{11}$ & -1 \\
  \bottomrule
\end{tabular}

\begin{mathpar}
  T(\ket{xy}\ket{-}) = (-1)^{x \land y} \ket{xy}\ket{-}
\end{mathpar}

Function value being part of the amplitude $\Rightarrow$ \emph{phased representation}.

\end{solution}

\end{questions}
\end{document}
